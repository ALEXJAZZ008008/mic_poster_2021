\documentclass[misc, color=UCLburgundy, margin=1cm]{uclposter}

\input{commands.tex}

\usepackage{bm}
\usepackage{algorithm}
\usepackage{algorithmic}
\usepackage{caption}
\usepackage{blindtext}
\usepackage{siunitx}

\usepackage[acronym, nomain]{glossaries}

% Define "long-s" key: 
\glsaddkey* {longs}% key 
{\glsentrylong{\glslabel}s}% default value 
{\glsentrylongs}% command analogous to \glsentrytext 
{\Glsentrylongs}% command analogous to \Glsentrytext 
{\glslongs}% command analogous to \glstext 
{\Glslongs}% command analogous to \Glstext 
{\GLSlongs}% command analogous to \GLStext

%% Define "short-s" key: 
\glsaddkey* {shorts}% key 
{\glsentryshort{\glslabel}s}% default value 
{\glsentryshorts}% command analogous to \glsentrytext 
{\Glsentryshorts}% command analogous to \Glsentrytext 
{\glsshorts}% command analogous to \glstext 
{\Glsshorts}% command analogous to \Glstext 
{\GLSshorts}% command analogous to \GLStext

\DeclareRobustCommand{\glss}[1]
{%
  \ifglsused{#1}{\glsshorts{#1}}{\glslongs{#1} (\glsshorts{#1})\glsunset{#1}}%
}

\DeclareRobustCommand{\Glss}[1]
{%
  \ifglsused{#1}{\Glsshorts{#1}}{\Glslongs{#1} (\glsshorts{#1})\glsunset{#1}}%
}

\newacronym{18F-FDG}{18F-FDG}{Fluorine-18 Fludeoxyglucose}
\newacronym{1D}{1D}{1-Dimensional}
\newacronym{2D}{2D}{2-Dimensional}
\newacronym{3D}{3D}{3-Dimensional}
\newacronym[longs={3-Dimensional Point Clouds}, shorts={3DPCs}]{3DPC}{3DPC}{3-Dimensional Point Cloud}
\newacronym{4D}{4D}{4-Dimensional}
% \newacronym{4DCT}{4DCT}{4-Dimensional Computed Tomography}
\newacronym{4DCT}{4DCT}{4D Computed Tomography}
\newacronym{AC}{AC}{Attenuation Corrected}
\newacronym{AD}{AD}{Affine Deformation}
\newacronym{AP}{AP}{Anterior Posterior}
\newacronym{ATP}{ATP}{Adenosine Triphosphate}
% \newacronym{AV-CCT}{AV-CCT}{Averaged CINE-Computed Tomography}
\newacronym{AV-CCT}{AV-CCT}{Averaged CINE-CT}
\newacronym{BE}{BE}{Bending Energy}
\newacronym{BFGS}{BFGS}{Broyden Fletcher Goldfarb Shanno}
\newacronym{CC}{CC}{Correlation Coefficient}
% \newacronym{CCT}{CCT}{CINE-Computed Tomography}
\newacronym{CCT}{CCT}{CINE-CT}
\newacronym{CG}{CG}{Conjugate Gradient}
\newacronym{COM}{COM}{Centre of Mass}
\newacronym[longs={Control Points}, shorts={CPs}]{CP}{CP}{Control Point}
\newacronym[longs={Control Point Grids}, shorts={CPGs}]{CPG}{CPG}{Control Point Grid}
\newacronym{CT}{CT}{Computed Tomography}
\newacronym{DDG}{DDG}{Data Driven Gating}
\newacronym{DD-PCA}{DD-PCA}{Data Driven Principal Component Analysis Surrogate Signal Extraction}
\newacronym{DD}{DD}{Data Driven}
\newacronym[longs={Deformation Vector Fields}, shorts={DVFs}]{DVF}{DVF}{Deformation Vector Field}
\newacronym{EANM}{EANM}{European Association of Nuclear Medicine}
\newacronym{EM}{EM}{Expectation Maximisation}
\newacronym{FDG}{FDG}{Fluorodeoxyglucose}
\newacronym{FFT}{FFT}{Fast Fourier Transform}
\newacronym[longs={Fields of View}, shorts={FOVs}]{FOV}{FOV}{Field Of View}
\newacronym{FWHM}{FWHM}{Full Width at Half Maximum}
\newacronym{GAN}{GAN}{Generative Adversarial Networks}
\newacronym{GD}{GD}{Gradient Descent}
\newacronym{GE}{GE}{General Electric}
\newacronym{GT}{GT}{Ground Truth}
\newacronym{HU}{HU}{Hounsfield Unit}
\newacronym[longs={Image Registrations}, shorts={IRs}]{IR}{IR}{Image Registration}
\newacronym{KBq/mL}{KBq/mL}{Kilo Becquerel per Millilitre}
\newacronym{KeV}{KeV}{Kilo Electron Volt}
\newacronym{KRG}{KRG}{Kinetic Respiratory Gating}
\newacronym{KV}{KV}{Kilo Volt}
\newacronym{L-BFGS-B}{L-BFGS-B}{Low memory Broyden Fletcher Goldfarb Shanno Bounded}
\newacronym{L-BFGS}{L-BFGS}{Low memory Broyden Fletcher Goldfarb Shanno}
\newacronym[longs={Light Emitting Diodes}, shorts={LEDs}]{LED}{LED}{Light Emitting Diode}
\newacronym{LE}{LE}{Linear Energy}
\newacronym{LR}{LR}{Linear Regression}
\newacronym[longs={Lines of Response}, shorts={LORs}]{LOR}{LOR}{Line of Response}
\newacronym{MAE}{MAE}{Mean Absolute Error}
\newacronym{MAD}{MAD}{Median Absolute Difference}
\newacronym{MAPE}{MAPE}{Mean Absolute Percentage Error}
\newacronym{MBF}{MBF}{Myocardial Blood Flow}
\newacronym{MCIR}{MCIR}{Motion Compensated Image Reconstruction}
\newacronym[longs={Motion Compensated Images}, shorts={MCIs}]{MCI}{MCI}{Motion Compensated Image}
\newacronym{MC}{MC}{Motion Correction}
\newacronym{MIC}{MIC}{Medical Imaging Convention}
\newacronym{MI}{MI}{Mutual Information}
\newacronym{ML}{ML}{Maximum Likelihood}
\newacronym{MLAA}{MLAA}{Maximum Likelihood Reconstruction of Activity and Attenuation}
\newacronym{MLE}{MLE}{Maximum Likelihood Estimation}
\newacronym{MLEM}{MLEM}{Maximum Likelihood Expectation Maximisation}
\newacronym[longs={Motion Models}, shorts={MMs}]{MM}{MM}{Motion Model}
\newacronym{MPI}{MPI}{Myocardial Perfusion Imaging}
\newacronym{MR}{MR}{Magnetic Resonance}
\newacronym{MSc}{MSc}{Master of Science}
\newacronym{MSE}{MSE}{Mean Squared Error}
\newacronym[longs={Attenuation Maps}, shorts={Mu-Maps}]{Mu-Map}{Mu-Map}{Attenuation Map}
\newacronym{NAC}{NAC}{Non-Attenuation Corrected}
\newacronym{NMI}{NMI}{Normalised Mutual Information}
\newacronym{ND}{ND}{n-Dimensional}
\newacronym{NMC}{NMC}{Non-Motion Corrected}
\newacronym{NN}{NN}{Neural Network}
\newacronym{NRD}{NRD}{Non-Rigid Deformation}
\newacronym{NTOF}{NTOF}{Non-Time-of-Flight}
\newacronym{OSEM}{OSEM}{Ordered Subset Expectation Maximisation}
\newacronym[longs={Principal Components}, shorts={PCs}]{PC}{PC}{Principal Component}
\newacronym{PCA}{PCA}{Principal Component Analysis}
\newacronym{PCC}{PCC}{Pearson Correlation Coefficient}
\newacronym{PET}{PET}{Positron Emission Tomography}
\newacronym{PLL}{PLL}{Poisson Log Likelihood}
\newacronym{PSD}{PSD}{Power Spectral Density}
\newacronym{PSMA}{PSMA}{Prostate Specific Membrane Antigen}
\newacronym{RANSAC}{RANSAC}{Random Sample Consensus}
\newacronym{RCM}{RCM}{Respiratory Correspondence Model}
\newacronym{RD}{RD}{Rigid Deformation}
\newacronym{RDP}{RDP}{Relative Difference Prior}
\newacronym{RM}{RM}{Respiratory Motion}
\newacronym{RMC}{RMC}{Respiratory Motion Correction}
\newacronym{RMSE}{RMSE}{Root Mean Square Error}
\newacronym[longs={Regions of Interest}, shorts={ROIs}]{ROI}{ROI}{Region of Interest}
\newacronym{RPM}{RPM}{Real Time Position Management}
\newacronym{RTPM}{RTPM}{Real Time Position Management}
\newacronym{SAM}{SAM}{Spectral Analysis Method}
\newacronym{SGD}{SGD}{Stochastic Gradient Descent}
\newacronym{SI}{SI}{Superior Inferior}
\newacronym{SIRF}{SIRF}{Synergistic Image Reconstruction Framework}
\newacronym{SNR}{SNR}{Signal to Noise Ratio}
\newacronym[longs={Surrogate Signals}, shorts={SSs}]{SS}{SS}{Surrogate Signal}
\newacronym{SSD}{SSD}{Sum of Squared Differences}
\newacronym{STFT}{STFT}{Short-time Fourier transform}
\newacronym{STIR}{STIR}{Software for Tomographic Image Reconstruction}
\newacronym{SUV}{SUV}{Standard Uptake Value}
\newacronym{SVD}{SVD}{Singular Value Decomposition}
\newacronym{TOF}{TOF}{Time-of-Flight}
\newacronym{TPS}{TPS}{Thin Plate Spline}
\newacronym{XCAT}{XCAT}{4-Dimensional Extended Cardiac Torso}

\glsunset{18F-FDG}
\glsunset{1D}
\glsunset{2D}
\glsunset{3D}
\glsunset{4D}
\glsunset{4DCT}
\glsunset{ATP}
\glsunset{BFGS}
\glsunset{CT}
\glsunset{EANM}
\glsunset{FDG}
\glsunset{GE}
\glsunset{L-BFGS-B}
\glsunset{L-BFGS}
\glsunset{LED}
\glsunset{MIC}
\glsunset{MLAA}
\glsunset{MLEM}
\glsunset{MR}
\glsunset{MSc}
\glsunset{Mu-Map}
\glsunset{NTOF}
\glsunset{OSEM}
\glsunset{PCA}
\glsunset{PET}
\glsunset{RANSAC}
\glsunset{RPM}
\glsunset{RTPM}
\glsunset{SIRF}
\glsunset{STIR}
\glsunset{SUV}
\glsunset{TOF}
\glsunset{XCAT}

\usepackage[style=ieee, maxbibnames=1, minbibnames=1, maxcitenames=1, mincitenames=1, backend=biber, defernumbers=false]{biblatex}
\addbibresource{./Biblio.bib}

\AtEveryBibitem{\clearfield{month}}
\AtEveryBibitem{\clearfield{day}}
\AtEveryBibitem{\clearfield{volume}}
\AtEveryBibitem{\clearfield{issue}}
\AtEveryBibitem{\clearfield{pages}}
\AtEveryBibitem{\clearfield{number}}
\AtEveryBibitem{\clearfield{title}}
\AtEveryBibitem{\clearfield{isbn}}
\AtEveryBibitem{\clearfield{keywords}}
\AtEveryBibitem{\clearfield{issn}}
\AtEveryBibitem{\clearfield{journal}}

\usepackage{fontspec}
\setmainfont[Ligatures=TeX]{LexendDeca-Regular.ttf}

\begin{document}
    \title{Comparison of Motion Correction Methods Incorporating Motion Modelling for PET/CT Using a Single Breath Hold Attenuation Map}
    
    \author[12*]{Alexander~C.~Whitehead}
    \author[2]{Ander~Biguri}
    \author[3]{Kuan-Hao~Su}
    \author[3]{Scott~D.~Wollenweber}
    \author[3]{Charles~W.~Stearns}
    \author[1]{Brian~F.~Hutton}
    \author[2]{\newline~Jamie~R.~McClelland}
    \author[12]{Kris~Thielemans}
    
    \affil[1]{INM, UCL}
    \affil[2]{CMIC, UCL}
    \affil[3]{GE Healthcare}
    \affil[*]{alexander.whitehead.18@ucl.ac.uk}
    
    \maketitle

    \begin{multicols}{4}
        \normalsize
        
        \section*{Introduction}
            \begin{highlightbox}[UCLlightgreen]
                \begin{itemize}
                    \item \gls{RM} reduces resolution and quantification accuracy in \gls{PET}.
                    \item A \gls{MM} is a \gls{MC} technique where \glss{DVF} are parameterised by a \gls{SS}~\cite{McClelland2013}.
                    \item A \gls{MM} is more robust to noise and allows for correction of unseen data.
                    \item This work will demonstrate a comparison of registration methods, pair-wise and group-wise, including \glss{MM}.
                    \item This work will fix the \gls{Mu-Map} at end inhalation. This is more clinically relevant but also challenging when compared to~\cite{Whitehead2020PET/CTFields}.
                    \item This work differentiates itself by using two \glss{SS}, and group-wise registration.
                \end{itemize}
            \end{highlightbox}
        
        \section*{Methods}
            \subsection*{\underline{\textbf{XCAT Volume Generation}}}
                \begin{itemize}
                    \item \gls{XCAT} generated 240 volumes using a \SI{120}{\second} respiratory trace.
                    \item Activity concentrations from a static \gls{18F-FDG} patient scan.
                    \item \gls{FOV} including the base of the lungs with a \SI{20}{\milli\metre} diameter lesion.
                \end{itemize}
            
            \subsection*{\underline{\textbf{PET Simulation and Reconstruction}}}
                \begin{itemize}
                    \item Simulated using the geometry of a \gls{GE} Discovery 710.
                    \item Pseudo-randoms and scatter were added.
                    \item A respiratory \gls{SS} was generated using \gls{PCA}~\cite{Thielemans2011}.
                    \item Gated into 30 respiratory bins using displacement gating, \gls{SS} and its gradient (10 amplitude and three gradient bins).
                    \item Reconstructed without \gls{AC} using \gls{OSEM} with two full iterations and 24 subsets.
                \end{itemize}
            
            \subsection*{\underline{\textbf{Registration}}}
                \begin{itemize}
                    \item Pre-processing including; replication of end-slices and Yeo-Johnson transform.
                    \item Two registration methods were used; pair-wise (reference position selected as the gate with the highest number of counts) and group-wise registration (initial pair-wise step).
                    \item B-spline registration using \gls{NMI}.
                    \item \acrlong{CPG} spacing, \acrlong{BE} weight and number of iterations tuned using a grid search.
                \end{itemize}
            
            \subsection*{\underline{\textbf{Motion Model Estimation}}}
                \begin{itemize}
                    \item \gls{MM} fit using weighted \gls{LR} between registration \glss{DVF} and two \glss{SS}
                    \item \gls{LR} weight set from total counts in gate
                    \item For group-wise registration, \gls{MM} fit between each iteration.
                \end{itemize}
            
            \subsection*{\underline{\textbf{Attenuation Map Warping}}}
                \begin{itemize}
                    \item \gls{Mu-Map} at end inhalation selected.
                    \item \gls{MC} \gls{PET} volume registered to \gls{Mu-Map}.
                    \item Resulting \glss{DVF} composed.
                    \item Inverse \glss{DVF} used to warp \gls{Mu-Map} to each gate.
                \end{itemize}
            
            \subsection*{\underline{\textbf{Image Reconstruction with AC}}}
                \begin{itemize}
                    \item Data re-reconstructed with \gls{AC} and \gls{MC} reapplied as above.
                    \item Volumes post-filtered with a Gaussian smoothing, (\gls{FWHM} of \SI{6.39}{\milli\metre} in transverse plane and \SI{3.27}{\milli\metre} in the axial direction).
                \end{itemize}
            
            \subsection*{\underline{\textbf{Evaluation}}}
                \begin{highlightbox}[UCLlightgreen]
                    \begin{itemize}
                        \item Data also reconstructed without \gls{MC}, using either a sum of all \glss{Mu-Map} or the end inhalation \gls{Mu-Map}.
                        \item Volumes without \gls{MC} registered to the position of the \gls{Mu-Map}.
                        \item \glss{DVF} generated by each method were also applied to noiseless data for visual analysis.
                        \item Comparisons used included; a profile over the lesion, \gls{SUV}\textsubscript{max} and \gls{SUV}\textsubscript{peak}.
                    \end{itemize}
                \end{highlightbox}
        
        \section*{Results}
            \begin{figure}[H]
                \centering
                \hbox{\hspace{-1.0cm} \includegraphics[width=1.1\linewidth]{visual_analysis.png}}
                \begin{highlightbox}[UCLlightblue]
                    \captionsetup{singlelinecheck=false, justification=centering}
                    \caption{First column \gls{AC} \gls{MC} reconstructions, second column noiseless data. Colour map ranges consistent for all images in each column.}
                \end{highlightbox}
            \end{figure}
            
            \begin{figure}[H]
                \centering
                \includegraphics[width=1.0\linewidth]{profile.png}
                \begin{highlightbox}[UCLlightblue]
                    \captionsetup{singlelinecheck=false, justification=centering}
                    \caption{Profile across the lesion.}
                \end{highlightbox}
            \end{figure}
            
            \vspace{-1.0cm}
            
            \begin{table}[H]
                \centering
                \begin{highlightbox}[UCLlightblue]
                    \captionsetup{singlelinecheck=false, justification=centering}
                    \caption{Comparison of \gls{SUV}\textsubscript{max} and \gls{SUV}\textsubscript{peak}.}
                \end{highlightbox}
                \vspace{1.0cm}
                \resizebox*{1.0\linewidth}{!}
                {
                    \begin{tabular}{||c|cc||}
                        \hline
                        \textbf{\gls{SUV}}                  & \textbf{Max}  & \textbf{Peak} \\
                        \hline
                        \textbf{No Motion}                  & 9.50        & 9.06 \\
                        \hline
                        \textbf{Ungated Static \gls{CT}}    & 5.25        & 5.15 \\
                        \textbf{Ungated \acrlong{AV-CCT}}       & 5.38        & 5.07 \\
                        \hline
                        \textbf{Pair-wise}                  & 4.21        & 3.92 \\
                        \textbf{Pair-wise \gls{MM}}         & 6.63        & 6.07 \\
                        \hline
                        \textbf{Group-wise}                 & 4.42        & 4.21 \\
                        \textbf{Group-wise \gls{MM}}        & 7.64        & 7.03 \\
                        \hline
                    \end{tabular}
                }
            \end{table}
        
        \section*{Conclusion}
            \begin{itemize}
                \item Adding a \gls{MM} to any \gls{MC} method improved the quality of volumes produced.
                \item For \gls{MC} to be successful, for very noisy data, \glss{MM} are required in practice.
                \item Future work will focus on testing on patient data and incorporating the method into an iterative \acrlong{IR} and \gls{MC} method.
            \end{itemize}
        
        \AtNextBibliography{\tiny}
        \printbibliography
    \end{multicols}
\end{document}
